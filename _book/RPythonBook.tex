\documentclass[]{book}
\usepackage{lmodern}
\usepackage{amssymb,amsmath}
\usepackage{ifxetex,ifluatex}
\usepackage{fixltx2e} % provides \textsubscript
\ifnum 0\ifxetex 1\fi\ifluatex 1\fi=0 % if pdftex
  \usepackage[T1]{fontenc}
  \usepackage[utf8]{inputenc}
\else % if luatex or xelatex
  \ifxetex
    \usepackage{mathspec}
  \else
    \usepackage{fontspec}
  \fi
  \defaultfontfeatures{Ligatures=TeX,Scale=MatchLowercase}
\fi
% use upquote if available, for straight quotes in verbatim environments
\IfFileExists{upquote.sty}{\usepackage{upquote}}{}
% use microtype if available
\IfFileExists{microtype.sty}{%
\usepackage{microtype}
\UseMicrotypeSet[protrusion]{basicmath} % disable protrusion for tt fonts
}{}
\usepackage[margin=1in]{geometry}
\usepackage{hyperref}
\hypersetup{unicode=true,
            pdftitle={R to Python},
            pdfborder={0 0 0},
            breaklinks=true}
\urlstyle{same}  % don't use monospace font for urls
\usepackage{natbib}
\bibliographystyle{apalike}
\usepackage{color}
\usepackage{fancyvrb}
\newcommand{\VerbBar}{|}
\newcommand{\VERB}{\Verb[commandchars=\\\{\}]}
\DefineVerbatimEnvironment{Highlighting}{Verbatim}{commandchars=\\\{\}}
% Add ',fontsize=\small' for more characters per line
\usepackage{framed}
\definecolor{shadecolor}{RGB}{248,248,248}
\newenvironment{Shaded}{\begin{snugshade}}{\end{snugshade}}
\newcommand{\KeywordTok}[1]{\textcolor[rgb]{0.13,0.29,0.53}{\textbf{#1}}}
\newcommand{\DataTypeTok}[1]{\textcolor[rgb]{0.13,0.29,0.53}{#1}}
\newcommand{\DecValTok}[1]{\textcolor[rgb]{0.00,0.00,0.81}{#1}}
\newcommand{\BaseNTok}[1]{\textcolor[rgb]{0.00,0.00,0.81}{#1}}
\newcommand{\FloatTok}[1]{\textcolor[rgb]{0.00,0.00,0.81}{#1}}
\newcommand{\ConstantTok}[1]{\textcolor[rgb]{0.00,0.00,0.00}{#1}}
\newcommand{\CharTok}[1]{\textcolor[rgb]{0.31,0.60,0.02}{#1}}
\newcommand{\SpecialCharTok}[1]{\textcolor[rgb]{0.00,0.00,0.00}{#1}}
\newcommand{\StringTok}[1]{\textcolor[rgb]{0.31,0.60,0.02}{#1}}
\newcommand{\VerbatimStringTok}[1]{\textcolor[rgb]{0.31,0.60,0.02}{#1}}
\newcommand{\SpecialStringTok}[1]{\textcolor[rgb]{0.31,0.60,0.02}{#1}}
\newcommand{\ImportTok}[1]{#1}
\newcommand{\CommentTok}[1]{\textcolor[rgb]{0.56,0.35,0.01}{\textit{#1}}}
\newcommand{\DocumentationTok}[1]{\textcolor[rgb]{0.56,0.35,0.01}{\textbf{\textit{#1}}}}
\newcommand{\AnnotationTok}[1]{\textcolor[rgb]{0.56,0.35,0.01}{\textbf{\textit{#1}}}}
\newcommand{\CommentVarTok}[1]{\textcolor[rgb]{0.56,0.35,0.01}{\textbf{\textit{#1}}}}
\newcommand{\OtherTok}[1]{\textcolor[rgb]{0.56,0.35,0.01}{#1}}
\newcommand{\FunctionTok}[1]{\textcolor[rgb]{0.00,0.00,0.00}{#1}}
\newcommand{\VariableTok}[1]{\textcolor[rgb]{0.00,0.00,0.00}{#1}}
\newcommand{\ControlFlowTok}[1]{\textcolor[rgb]{0.13,0.29,0.53}{\textbf{#1}}}
\newcommand{\OperatorTok}[1]{\textcolor[rgb]{0.81,0.36,0.00}{\textbf{#1}}}
\newcommand{\BuiltInTok}[1]{#1}
\newcommand{\ExtensionTok}[1]{#1}
\newcommand{\PreprocessorTok}[1]{\textcolor[rgb]{0.56,0.35,0.01}{\textit{#1}}}
\newcommand{\AttributeTok}[1]{\textcolor[rgb]{0.77,0.63,0.00}{#1}}
\newcommand{\RegionMarkerTok}[1]{#1}
\newcommand{\InformationTok}[1]{\textcolor[rgb]{0.56,0.35,0.01}{\textbf{\textit{#1}}}}
\newcommand{\WarningTok}[1]{\textcolor[rgb]{0.56,0.35,0.01}{\textbf{\textit{#1}}}}
\newcommand{\AlertTok}[1]{\textcolor[rgb]{0.94,0.16,0.16}{#1}}
\newcommand{\ErrorTok}[1]{\textcolor[rgb]{0.64,0.00,0.00}{\textbf{#1}}}
\newcommand{\NormalTok}[1]{#1}
\usepackage{longtable,booktabs}
\usepackage{graphicx,grffile}
\makeatletter
\def\maxwidth{\ifdim\Gin@nat@width>\linewidth\linewidth\else\Gin@nat@width\fi}
\def\maxheight{\ifdim\Gin@nat@height>\textheight\textheight\else\Gin@nat@height\fi}
\makeatother
% Scale images if necessary, so that they will not overflow the page
% margins by default, and it is still possible to overwrite the defaults
% using explicit options in \includegraphics[width, height, ...]{}
\setkeys{Gin}{width=\maxwidth,height=\maxheight,keepaspectratio}
\IfFileExists{parskip.sty}{%
\usepackage{parskip}
}{% else
\setlength{\parindent}{0pt}
\setlength{\parskip}{6pt plus 2pt minus 1pt}
}
\setlength{\emergencystretch}{3em}  % prevent overfull lines
\providecommand{\tightlist}{%
  \setlength{\itemsep}{0pt}\setlength{\parskip}{0pt}}
\setcounter{secnumdepth}{5}
% Redefines (sub)paragraphs to behave more like sections
\ifx\paragraph\undefined\else
\let\oldparagraph\paragraph
\renewcommand{\paragraph}[1]{\oldparagraph{#1}\mbox{}}
\fi
\ifx\subparagraph\undefined\else
\let\oldsubparagraph\subparagraph
\renewcommand{\subparagraph}[1]{\oldsubparagraph{#1}\mbox{}}
\fi

%%% Use protect on footnotes to avoid problems with footnotes in titles
\let\rmarkdownfootnote\footnote%
\def\footnote{\protect\rmarkdownfootnote}

%%% Change title format to be more compact
\usepackage{titling}

% Create subtitle command for use in maketitle
\newcommand{\subtitle}[1]{
  \posttitle{
    \begin{center}\large#1\end{center}
    }
}

\setlength{\droptitle}{-2em}

  \title{R to Python}
    \pretitle{\vspace{\droptitle}\centering\huge}
  \posttitle{\par}
    \author{}
    \preauthor{}\postauthor{}
    \date{}
    \predate{}\postdate{}
  
\usepackage{booktabs}
\usepackage{amsthm}
\makeatletter
\def\thm@space@setup{%
  \thm@preskip=8pt plus 2pt minus 4pt
  \thm@postskip=\thm@preskip
}
\makeatother

\usepackage{amsthm}
\newtheorem{theorem}{Theorem}[chapter]
\newtheorem{lemma}{Lemma}[chapter]
\theoremstyle{definition}
\newtheorem{definition}{Definition}[chapter]
\newtheorem{corollary}{Corollary}[chapter]
\newtheorem{proposition}{Proposition}[chapter]
\theoremstyle{definition}
\newtheorem{example}{Example}[chapter]
\theoremstyle{definition}
\newtheorem{exercise}{Exercise}[chapter]
\theoremstyle{remark}
\newtheorem*{remark}{Remark}
\newtheorem*{solution}{Solution}
\begin{document}
\maketitle

{
\setcounter{tocdepth}{1}
\tableofcontents
}
\begin{Shaded}
\begin{Highlighting}[]
\KeywordTok{install.packages}\NormalTok{(}\StringTok{"bookdown"}\NormalTok{,}\StringTok{"reticulate"}\NormalTok{)}
\CommentTok{# or the development version}
\CommentTok{# devtools::install_github("rstudio/bookdown")}
\end{Highlighting}
\end{Shaded}

\chapter{Front Material}\label{front-material}

\section{Colophon}\label{colophon}

R to Python

Thoughts of an R programmer Learning Python

\begin{figure}
\centering
\includegraphics{Janus-iStock-494706611alpha.png}
\caption{Janus}
\end{figure}

by David A York

Copyright 2018 David A York

\url{http://crunches-data.appspot.com/contact.html}

This manuscript may be freely copied and distributed, under the MIT
Licence

self-published, on Toth-York Imprint, Calhoun GA USA, November 27, 2018

\url{https://github.com/medmatix/RtoPythonThoughts}

\begin{center}\rule{0.5\linewidth}{\linethickness}\end{center}

\begin{quote}
``In ancient Roman religion and myth, Janus (/ˈdʒeɪnəs/; Latin: IANVS
(Iānus), pronounced {[}ˈjaː.nus{]}) is the god of beginnings, gates,
transitions, time, duality, doorways,{[}1{]} passages, and endings. He
is usually depicted as having two faces, since he looks to the future
and to the past.''
\end{quote}

\begin{quote}
from Wikipedia, \url{https://en.wikipedia.org/wiki/Janus}, accessed 27
November, 2018 at 4:17PM
\end{quote}

\section{Preface}\label{preface}

This book is a work in progress on the thoughts of an R programmer (the
author) moving to (or learning) Python. It is intended as a comparison
of R and Python for those already using R. The hop is to ease the way
for those in particular newer to programming as opposed to the quick
task oriented scripting which R lends itself so well to.

As with any language the most visible presence on line are the hardened
champions of the languages with the pressure to standardization (not a
bad thing per-se) and dogmatic adherence to the ``blah blah'' way of
doing things (which is desirable I think).

Having started as an R user, then as an R programmer, I have relatively
recently embarked on learning python as well. Why I would do this is
rather complicated to explain, particularly as I had knowledge of other
general programming languages, which would conceivably serve such a
role, particularly BASIC and Java. Suffice to say that as a budding Data
Scientist I see lots of reasons to know both R and Python. BASIC, though
still extant it is very limited in support and relevant libraries by
comparison. Java . . . well, as a strong open source advocate I feel
Java is bound to Oracle in many ways for all time. It is likely they
have no intention of truly releasing it into the public domain, ever.
There is far more rationel to learn C++ these days than Java for the
sake of what is arguably only minor difference in learning curve.

I absolutely love R! When I returned to school to study Applied Math and
Statistics it was an ever present partner for me. It could do most
things I needed, though matrix math was less smooth than Matlab the
price was right (not withstanding Octave or Scilab). However, I
increasingly found the need for a general purpose programming language
and Python was there growing in the guise of `the' data science
alternative.

But while R was great with vectorized functions python, at it's core,
was weak in this regard. The libraries for python have been growing
steadily in number and variety, and the functionality gap between it and
R narrows in cluding for vectorized application. I still cannot see a
Data Scientist being as productive not knowing R at this point. However,
I see no reason not to also know Python - ergo, I think any serious Data
Science needs to know both; AND, there is the added incentive of the
interoperability of R and Python (and C++)! From R we have
\href{https://cran.cnr.berkeley.edu/web/packages/XRPython/index.html}{XRPython}
and
\href{https://cran.cnr.berkeley.edu/web/packages/reticulate/index.html}{reticulate}
and from Python we have \href{https://rpy2.bitbucket.io/}{Rpy2}.

The Organization of the book is in 2 parts. A close Comparison and
discussion of R and Python unburdened with a mandate to teach beginning
programming in either. The is overview of part 1, I only touch quite
generally on the nature of the usual programming sturctures. Variables,
datatypes and data structures, binary decisions, repeating code blocks
and so are part of all declarative or functional programming language -
the rest is just details. The Same can be side for laguages introducing
object-oriented programming models.

In the second part of the book, an opportunity is taken to compare how
data science is done in R and Python in the various domains we work.

One could consider this as a second source for R users learning python,
after the beginner courses to get the python syntax vocabulary.

\chapter{Introduction}\label{introduction}

Learning a new programming or scripting language begs comparison. Often
this is not overly helpful to the learner who, coming from a zone of
comfort is often frustrated by old subtleties not consciously
recognized, being yet unknown in the new. I recall a family member,
completely new to computers, saying why doesn't it know what I mean
expecting the implied parts of language to transfer from English (or
German, or Chinese etc.) to computer language.

R writers early on recognize the absence of vectorization of functions
in core python. It's available though in numpy and pandas, but base
python doesn't have it and they miss it. You'll get there, really.
Object Orientation is also not natural to either R or Python. It came
later to both, but it seems more smoothly accomplished by python than R.

I soon realized, though that you had to know the programming
commonalities of python first; R comparisons really should start with
the libraries of python. R in someways has cut right to the data science
chase often at the expense of strong programming constructs. This is the
nod to interactive coding and scripting rather than the need to
undertake formal programming. Python began early on with a scripting
functionality but it wasn't until the development of iPython the
interactive python could be realized in the way that R users were used
to.

All this is not to say that R was lacking in basic programming
constructs from the start, it wasn't; but he proportion of users who
used R in a quick-and-dirty small problem focused way was very high.
This suited it's use in a statistical learning environment. I should
think that those who try to start python at the same time as their first
statistics course will find the learning curve too steep to help with
the statistics part of the need.

\part{R and Python, Side by
Side}\label{part-r-and-python-side-by-side}

\chapter{A Comparative Look}\label{a-comparative-look}

As already mentioned, it is not the intent of this treatise to teach
either beginning R or python scripting. There are many good sources for
this and yet another beginners book or tutorial is, I believe, not
needed. For ease of reference I include
\href{https://github.com/medmatix/RtoPythonThoughts/blob/master/Appendix_1.md}{Appendix1}
as an overview of relevant R and python syntax for the general
programming constructs any functional language. In Part 1, at the risk
of being pedantic, I am simply giviing a fairly generic overview of the
components of the typical script or program as a framework to be used in
organizing our R and Python comparisons.

\section{Functional Programming
Commonalities.}\label{functional-programming-commonalities.}

In functional programming operational code is incorporated by task and
purpose into blocks of code which are called by name in the body of a
script. This usually includes some connecting statemts and assignments
which cause the functions to operate together to accomplish some larger
task.

Historically, Fortran and Basic code in common accademic and educational
use, became unwieldy as users became, more and more, writers of rampant
spagetti code.The introduction of the concept of `structured
programming' and the uniform reliance on subroutine and function call
tamed the spagetti-monster. Extension to functional programming with
pascal and modern C was a natural extension of this trend.

So the reliance of R and Python on script collections of functions
calling each other is what we are at now. Functions are subprograms that
have all the statement tasks common to most programs. I will deal in a
general way with these statement types and include comparsions between
how R and Python each express the tasks. It is not the intent of this
book to actually teach beginning programming as previously mentioned.
Consult the bibliography for some suggested sources for peginning
programmers.

The later innovation of object oriented programming concepts evolving
from this will be touched on in a comparative fashion in another
chapter.

\subsection{Statements and
Expressions}\label{statements-and-expressions}

A program or script is of course considered as a series of statement
stored together. These statements constitute expressions of mathemetical
equations (see specifically \href{}{Chapter 5}), choices, and other
orders to be carried out by the machine.

With any program or script statements tasks and values stored are
executed or accessed from memory locations. This was the innovation
conceived of by John Von Neumann which moved programming from the moving
wire jumpers to the input to the machine from various codes punched into
tapes or cards to be held in memory while being used.

\subsection{Assignment of Values to Memory and the
Variable}\label{assignment-of-values-to-memory-and-the-variable}

Variables are memory locations assigned values as the computing process
goes on. The classification of specific locations determines a type of
data held there.

The codes stored in the machine include a specific operator code to
direct the move of a value received for input or a statement's result to
a memory location. This is called and assignment.

Traditionally R has used the combined dash-less-than symbols
`-\textgreater{}' for this purpose but the simple equals sign `=' now
also works in this fashion. Python uses the same equals sign for
assignments.

Variables are named in both R and Python and neither may begin with a
digit. There are other restrictions and conventions which are not gone
into here.

\subsection{Decisions and Choices}\label{decisions-and-choices}

As currently implemented all choices by a computer resolve to yes or no
results. Some semblance or the greys we think in are a result of a
collection or range of yes-no questions. These choices are explicit of
implicit `if' queries.

\subsection{Doing it Over Again (and again
\ldots{})}\label{doing-it-over-again-and-again}

Loops are created within a program logic, usually involving a choice
test to be executed which cause tasks to be repeated zero or more times.
It is considered bad parctice to create a loop without some form of exit
(short of having to power down the machine).

\subsection{Functions}\label{functions}

Functions in fortran were short program clips that did a task and
immediately returned a value . Subroutines are more like what functions
in R and Python are, though returning to the calling place with some
value is a frequent option with these functions as well. Functions may
provide a new programming element like a square root (python,
math.sqrt(x)) extending the language, or functions may carry out a whole
task like carring out a liner regression, storing objects of the answer
in memory for retrieval, or return as an object explicitly. Ofter a
function is called that builds a whole grphic plot for display or other
output when called.

In R,

\begin{verbatim}
function.name <- function(arguments) 
{
  computations on the arguments
  maybe other code too
}
\end{verbatim}

and in Python indents take the place of brackets to define code blocks.
Self in the function spec is the calling code block, and if omitted, is
created by python implicitly. So, your first argument one way of another
is the calling namespace. Self in the indented code block is the
function itself.

\begin{verbatim}
def aroutine(self, args)
  computations on args
  maybe other code too
\end{verbatim}

The notation of R is C (and Java) - like and Python is Pascal-like in
format. Unlike pascal there is no `begin:' keyword required at anytime
in the script.

Functional programming with R and Python, will be expanded on in it's
own chapter later on.

\subsection{Objects and Classes}\label{objects-and-classes}

In general terms we will come to consider all program elements as
objects. A Class will be considered as a pattern definition for an
object. All objects are used as instances of some previously defined
class.

\chapter{Set-up for R and Python
Exercises}\label{set-up-for-r-and-python-exercises}

Clearly you have to get R and python onto your system. I have a bias
about depending on any en block installed, like Anaconda etc. Anaconda
is convenient but is too easy to loose locations of programs and tools
and get conflicts with installs outside of the grouped installers like
different Python destibutions and difficult to manage path complexity.

Just the same, Anaconda is a good choice if you are doing a clean
install and intend to do everthing inside that framework.Then installing
\href{https://www.anaconda.com/}{Anaconda,Python Data Science Platform}
can make things much easier in the long run for integrating R and Python
and Reproducible Research work.

For my own purposes, I install each separately. Start with R, and then
install Python through the Next Notebook.Finally install .

All of these sites will help you approach installs and configurations in
a variety of ways. As an R programmer first I installed R, then
\href{https://www.rstudio.com/}{RStudio}. Once installed, RStudio is a
convenient way to install the R packages as well and then Python 3
at\href{https://www.python.org/}{Python Software Foundation}. From that
point on use PIP package manager for python and RGUI for package
installs for python and R respectively and install jupyter see
\href{http://jupyter.org/install}{Project Jupyter}. This gave the
maximum flexibility for R and Python in their own environments.

I did do a parallel Anaconda install with R and RStudio and iPython for
Jupyter Notebooks but as mentioned there was some path confusion with
this approach. It does set you up nicely for any approach to data
science and development you could choose down the road.

You really are going to need an IDE for Python.
\href{https://www.spyder-ide.org/}{Spyder}, available separately or
installed through Anaconda is a solid python focused choice. I use
\href{https://www.eclipse.org/}{Eclipse} as you can use it for multiple
languages including R (not as strong yet) and python.

\section{Package Libraries}\label{package-libraries}

The strength of both R and Python is from their libraries.
\href{https://cran.cnr.berkeley.edu/index.html}{CRAN Package Repository}
and \href{http://www.bioconductor.org/about/}{Bioconnector} are both
well stocked with general and special purposed functions for R. RStudio
develops \href{https://www.rstudio.com/products/rpackages/}{R Packages}
integrating well with the RStudio environment. The are lost of R
packages in development awaiting addition to CRAN or hosted outside of
CRAN such as on \href{https://github.com/search?q=R+language}{Github}.

If the reader has spent any time with R at all you have likely made
inroads into these package sources. So how does Python compare? Python
comes with a large standard library providing ``out of the gate''
functionality to compare with R-core. The best handing of the
intricacies of the python standard library I've found is Doug Helmann's
\href{https://amazon.com/Python-Standard-Library-Example-Developers/dp/0134291050/}{The
Python 3 Standard Library by example}. Python Software foundation hosts
the \href{https://pypi.org/}{Python Package Index , PyPi} analygous to
the CRAN repositories and Bioconnector. Also, the most important
resource to get you up to speed is \href{https://www.scipy.org/}{SciPy
tools} consisting of SciPy, Numpy, Pandas and Matplotlib. These are the
essential minimum to do any R-like work in python. These and more can be
accessed as well through the
\href{https://numfocus.org/sponsored-projects}{Numfocus Projects}.
Finally, you have to check out the
\href{https://www.statsmodels.org/stable/index.html}{StatsModels}
package site.

\section{Pulling it Together}\label{pulling-it-together}

Having apprised ourselves of the breadth and availability of these
resources we will go forward and organize some of the ideas. R packages
can be installed with the R GUI but the RStudio environment is much
easier for this. Once a package is installed it is available in all
environments I've alluded to. Compiling and installing is always best
and I thing RStudio makes this fairly painless. Remember to start the
GUI, Anaconda or RStudion environment as administrator to keep things
together.

\subsection{The Prompts}\label{the-prompts}

When accessing the R and Python interpreters from the command shell (or
bash) prompt,

\begin{verbatim}
$
\end{verbatim}

after the version and copyright banners, if present, you are at an
interpreter prompt. For R this is:

\begin{verbatim}
>
\end{verbatim}

for python,

\begin{verbatim}
>>>
\end{verbatim}

for iPython

\begin{verbatim}
In [1]:
\end{verbatim}

from this point you enter what ever commands or statements you wish to
execute. These are executed immediately, or continued with continuation
prompts until executable. Once executed any declarations and output are
retained if the were assigned to some object to be held in memory.
Immediate answers are generally lost.

Otherwise, one groups statements into a file to be read all at once and
executed as a group. Objects created are again retained in memory by
default. This is a script of program. Script file extensions are
standardized for the operating system to recognize, eg.(.R for R, and
.py or pyc for python). Python scripts intended to be called by other
scripts for library code to be incorporated are called modules and end
in the same .py extension. A package can be considered, in both R and
python) as a collection of declarations, functions, classes, or modules
for library purposes.

\subsection{Installing Packages}\label{installing-packages}

From the R GUI prompt we have,

\begin{verbatim}
  package-install()
\end{verbatim}

Python packages from any source mentioned above can be installed with
the PIP utility. This can be done from the shell prompt (remember to
start as administrator or root):

\begin{verbatim}

  $ pip install matplotlib  
    # OR
  $ python -m pip install matplotlib
\end{verbatim}

This can be done as root with the anaconda prompt as well.

\subsection{Using Packages}\label{using-packages}

In R we must make sure the package (once installed) is loaded for our
session. This is done with the library() function,

\begin{Shaded}
\begin{Highlighting}[]
  \KeywordTok{library}\NormalTok{(}\StringTok{"reticulate"}\NormalTok{)}
\end{Highlighting}
\end{Shaded}

while in python we import the package previously installed as,

\begin{verbatim}
  import numpy
\end{verbatim}

Once imported or loaded into either, we can access the functions of MASS
directly by name. If we have a previous function of the same name it
gets overwritten unless in python we keep the name spaces separate. This
can be done by the form of the import statement. We can also import only
slected functions from a package module.

\begin{verbatim}

  import numpy as np             # usual format to use namespace segregation of functions
  from numpy import sum, matrix  # economical import, still can conflick without use of as
  
\end{verbatim}

\subsection{Using R and Python
Together}\label{using-r-and-python-together}

There are several ways that one could find helpful to use R and Python
together, to take best advantage of their respective strengths

\subsubsection{Working in R and Calling
Python}\label{working-in-r-and-calling-python}

In this instance one would find working in R the necessary starting
place and find the need to employ python functions or packages helpful
to your needs. A call would be made to the python function from R to
have a task completed by python and control returned to the R program.

\begin{Shaded}
\begin{Highlighting}[]
\KeywordTok{cat}\NormalTok{(}\StringTok{"PI in R is"}\NormalTok{, pi)}
\end{Highlighting}
\end{Shaded}

\begin{verbatim}
## PI in R is 3.141593
\end{verbatim}

\begin{Shaded}
\begin{Highlighting}[]
\ImportTok{import}\NormalTok{ math}
\BuiltInTok{print}\NormalTok{(math.pi)}
\end{Highlighting}
\end{Shaded}

\begin{verbatim}
## 3.141592653589793
\end{verbatim}

\subsubsection{Working in Python and Calling
R}\label{working-in-python-and-calling-r}

Working in python and calling an R object would look like,

\begin{verbatim}
# Accessing an R object from from python 
# (example from: http://rpy.sourceforge.net/rpy2/doc-dev/html/introduction.html)
import rpy2.robjects as robjects
rpi = robjects.r['pi']
print("PI from R = ", rpi)
\end{verbatim}

\subsubsection{Working in Jupyter and Writing in R, Python or Both in
the Same
Notebook}\label{working-in-jupyter-and-writing-in-r-python-or-both-in-the-same-notebook}

There are multiple kernel plugins available for jupyter notebooks. As A
rule, the notebooks only use one language kernel at a time. An
excepti9on is the SOS kernel.

\subsubsection{Working in rmarkdown and Using R and Python Code
Chunks}\label{working-in-rmarkdown-and-using-r-and-python-code-chunks}

Note: I have found behavior is unpredictible to call R objects back with
Python chunks run in rmarkdown (ie. inside another instance of R). The
identity of R\_User for rpy2 gets lost. There are limites to how deeply
you should nest languages within languages.

That said, the rmarkdown chuncks look like this.

\begin{verbatim}
'''{r}
cat("PI in R is", pi)
'''
\end{verbatim}

\begin{verbatim}
'''{python}
import math
print(math.pi)
'''
\end{verbatim}

\chapter{Basic Mathematics in R and
Python}\label{basic-mathematics-in-r-and-python}

This is pretty straight forward and intuitive stuff. Just the same, for
completeness here it is. The Standard libraries of python have the
function beyond the four arithmetic operations.

\section{Add, Subtract, Multiply and
Divide}\label{add-subtract-multiply-and-divide}

The usual quick calculation would be done in immediate or interactive
mode without relying on print statements. There is outwardly no
difference if the print is used as would be the case in a script.
However Printing a set of values reqires the set be organized into an
object in itself, a vector.

\begin{Shaded}
\begin{Highlighting}[]
\CommentTok{# Loading reticulate package to bring python interpreter on line.}
\KeywordTok{library}\NormalTok{(}\StringTok{"reticulate"}\NormalTok{)}
\end{Highlighting}
\end{Shaded}

\subsection{R Scripting}\label{r-scripting}

\begin{Shaded}
\begin{Highlighting}[]
\CommentTok{# Some immediate calculations, ie. using R like a calculator}
\DecValTok{2} \OperatorTok{+}\StringTok{ }\DecValTok{3}           \CommentTok{# (simple interactive) addition}
\end{Highlighting}
\end{Shaded}

\begin{verbatim}
## [1] 5
\end{verbatim}

\begin{Shaded}
\begin{Highlighting}[]
\DecValTok{5} \OperatorTok{-}\StringTok{ }\DecValTok{2}           \CommentTok{# subtraction}
\end{Highlighting}
\end{Shaded}

\begin{verbatim}
## [1] 3
\end{verbatim}

\begin{Shaded}
\begin{Highlighting}[]
\DecValTok{6} \OperatorTok{*}\StringTok{ }\DecValTok{2}           \CommentTok{# multiplication}
\end{Highlighting}
\end{Shaded}

\begin{verbatim}
## [1] 12
\end{verbatim}

\begin{Shaded}
\begin{Highlighting}[]
\DecValTok{4} \OperatorTok{/}\StringTok{ }\DecValTok{3}           \CommentTok{# division}
\end{Highlighting}
\end{Shaded}

\begin{verbatim}
## [1] 1.333333
\end{verbatim}

\begin{Shaded}
\begin{Highlighting}[]
\KeywordTok{print}\NormalTok{(}\DecValTok{4} \OperatorTok{/}\StringTok{ }\DecValTok{3}\NormalTok{)    }\CommentTok{# calculation in a script requires print to display an answer}
\end{Highlighting}
\end{Shaded}

\begin{verbatim}
## [1] 1.333333
\end{verbatim}

\begin{Shaded}
\begin{Highlighting}[]
\CommentTok{# Variable assignment and printing grouped variables as a vector}
\NormalTok{a <-}\StringTok{ }\NormalTok{(}\DecValTok{2} \OperatorTok{+}\StringTok{ }\DecValTok{3}\NormalTok{)           }\CommentTok{# addition}
\NormalTok{b <-}\StringTok{ }\NormalTok{(}\DecValTok{5} \OperatorTok{-}\StringTok{ }\DecValTok{2}\NormalTok{)           }\CommentTok{# subtraction}
\NormalTok{c <-}\StringTok{ }\NormalTok{(}\DecValTok{6} \OperatorTok{*}\StringTok{ }\DecValTok{2}\NormalTok{)           }\CommentTok{# multiplication}
\NormalTok{d <-}\StringTok{ }\NormalTok{(}\DecValTok{4} \OperatorTok{/}\StringTok{ }\DecValTok{3}\NormalTok{)           }\CommentTok{# division}
\KeywordTok{print}\NormalTok{ (}\KeywordTok{c}\NormalTok{(a, b, c, d))  }\CommentTok{# printing all results together one statement}
\end{Highlighting}
\end{Shaded}

\begin{verbatim}
## [1]  5.000000  3.000000 12.000000  1.333333
\end{verbatim}

Python, on the otherhand considers literal expressions and variables
individually (though we'll see they can be groups as well.) This thr
puthon print statement is by default printing one or more individual
objects as scalars whereas R print function prints object and there is
no scalar. The simplest object in R is the vector.

\subsection{Python Scripting}\label{python-scripting}

The simplest object in python is the scalar, having one of five
\textbf{basic datatypes},

\begin{itemize}
\tightlist
\item
  Integers
\item
  Floating-Point Numbers
\item
  Complex Numbers
\item
  Strings
\item
  Boolean Type
\end{itemize}

\begin{Shaded}
\begin{Highlighting}[]
\CommentTok{# The same immediate calculations, ie. using iPython like a calculator}
\BuiltInTok{print}\NormalTok{(}\DecValTok{2} \OperatorTok{+} \DecValTok{5}\NormalTok{, }\DecValTok{3} \OperatorTok{*} \DecValTok{8}\NormalTok{, }\DecValTok{5} \OperatorTok{-} \DecValTok{1}\NormalTok{, }\DecValTok{67} \OperatorTok{/} \DecValTok{3}\NormalTok{) }\CommentTok{# printing all immediate results together, one statement}
\end{Highlighting}
\end{Shaded}

\begin{verbatim}
## 7 24 4 22.333333333333332
\end{verbatim}

\begin{Shaded}
\begin{Highlighting}[]
\CommentTok{# OR assignment and print}
\NormalTok{a }\OperatorTok{=} \DecValTok{2} \OperatorTok{+} \DecValTok{5}           \CommentTok{# assigned addition}
\NormalTok{b }\OperatorTok{=} \DecValTok{5} \OperatorTok{-} \DecValTok{1}           \CommentTok{# assigned subtraction}
\NormalTok{c }\OperatorTok{=} \DecValTok{3} \OperatorTok{*} \DecValTok{8}           \CommentTok{# assigned multiplication}
\NormalTok{d }\OperatorTok{=} \DecValTok{67} \OperatorTok{/} \DecValTok{3}          \CommentTok{# assigned division}
\BuiltInTok{print}\NormalTok{(a,b,c,d)      }\CommentTok{# printing all results together one statement}
\end{Highlighting}
\end{Shaded}

\begin{verbatim}
## 7 4 24 22.333333333333332
\end{verbatim}

\begin{Shaded}
\begin{Highlighting}[]
\BuiltInTok{print}\NormalTok{(}\StringTok{'a ='}\NormalTok{, a, }\StringTok{'b ='}\NormalTok{,b, }\StringTok{'c ='}\NormalTok{,c, }\StringTok{'d ='}\NormalTok{,d) }\CommentTok{# annotating print is just using 4 literals and 4 variables}
\end{Highlighting}
\end{Shaded}

\begin{verbatim}
## a = 7 b = 4 c = 24 d = 22.333333333333332
\end{verbatim}

This differs from R where a single object is printed but that object may
be a group of objects combined.

\section{Other basic algebraic operators in R and
Python}\label{other-basic-algebraic-operators-in-r-and-python}

Next lets compare modulus, powers and interger division in R and python,

\begin{Shaded}
\begin{Highlighting}[]
\DecValTok{35} \OperatorTok\StringTok{ }\DecValTok{2}     \CommentTok{# modulo operator}
\end{Highlighting}
\end{Shaded}

\begin{verbatim}
## [1] 1
\end{verbatim}

\begin{Shaded}
\begin{Highlighting}[]
\DecValTok{35} \OperatorTok\StringTok{ }\DecValTok{2}    \CommentTok{# integer division}
\end{Highlighting}
\end{Shaded}

\begin{verbatim}
## [1] 17
\end{verbatim}

\begin{Shaded}
\begin{Highlighting}[]
\DecValTok{35}\OperatorTok{^}\DecValTok{2}        \CommentTok{# powers in R\textbackslash{}}
\end{Highlighting}
\end{Shaded}

\begin{verbatim}
## [1] 1225
\end{verbatim}

\begin{Shaded}
\begin{Highlighting}[]
\BuiltInTok{print}\NormalTok{(}\DecValTok{35}\OperatorTok{%}\DecValTok{2}\NormalTok{)       }\CommentTok{# modulo operator}
\end{Highlighting}
\end{Shaded}

\begin{verbatim}
## 1
\end{verbatim}

\begin{Shaded}
\begin{Highlighting}[]
\BuiltInTok{print}\NormalTok{(}\DecValTok{35} \OperatorTok{//} \DecValTok{2}\NormalTok{)    }\CommentTok{# integer division       * note difference from R}
\end{Highlighting}
\end{Shaded}

\begin{verbatim}
## 17
\end{verbatim}

\begin{Shaded}
\begin{Highlighting}[]
\BuiltInTok{print}\NormalTok{(}\DecValTok{35}\OperatorTok{**}\DecValTok{2}\NormalTok{)      }\CommentTok{# powers in python     * note difference from R}
\end{Highlighting}
\end{Shaded}

\begin{verbatim}
## 1225
\end{verbatim}

Note here with python the print statements are required otherwise only
the last immediate calculation would get output to text.

When we extend our algebraic calculations in R the usual functions of
square root, absolute value and exponentiation we still haven't had to
load any library packages. All these are part of the core R.

\begin{Shaded}
\begin{Highlighting}[]
\KeywordTok{abs}\NormalTok{(}\OperatorTok{-}\DecValTok{5}\NormalTok{)          }\CommentTok{# absolute value}
\end{Highlighting}
\end{Shaded}

\begin{verbatim}
## [1] 5
\end{verbatim}

\begin{Shaded}
\begin{Highlighting}[]
\KeywordTok{sqrt}\NormalTok{(}\DecValTok{16}\NormalTok{)         }\CommentTok{# square root function}
\end{Highlighting}
\end{Shaded}

\begin{verbatim}
## [1] 4
\end{verbatim}

\begin{Shaded}
\begin{Highlighting}[]
\KeywordTok{exp}\NormalTok{(}\DecValTok{0}\NormalTok{)           }\CommentTok{# exponent (e^0 etc)}
\end{Highlighting}
\end{Shaded}

\begin{verbatim}
## [1] 1
\end{verbatim}

Not necessarily so for python. We still have the basic functions for
algebraic tasks but from sqrt on ward, we now need to go to the standard
math library for these. As an aside we could have got more powerful
versions of these functions (and more) importing numpy or numba or scipy
library packages. This will come up in more detail later.

\begin{Shaded}
\begin{Highlighting}[]
\BuiltInTok{print}\NormalTok{(}\BuiltInTok{abs}\NormalTok{(}\OperatorTok{-}\DecValTok{5}\NormalTok{))}
\end{Highlighting}
\end{Shaded}

\begin{verbatim}
## 5
\end{verbatim}

\begin{Shaded}
\begin{Highlighting}[]
\BuiltInTok{print}\NormalTok{(}\BuiltInTok{divmod}\NormalTok{(}\DecValTok{10}\NormalTok{,}\DecValTok{3}\NormalTok{))             }\CommentTok{# Returns quotient and remainder of integer division}
\end{Highlighting}
\end{Shaded}

\begin{verbatim}
## (3, 1)
\end{verbatim}

\begin{Shaded}
\begin{Highlighting}[]
\BuiltInTok{print}\NormalTok{(}\BuiltInTok{max}\NormalTok{(}\DecValTok{2}\NormalTok{,}\DecValTok{10}\NormalTok{,}\DecValTok{3}\NormalTok{,}\DecValTok{14}\NormalTok{,}\DecValTok{4}\NormalTok{,}\DecValTok{28}\NormalTok{,}\DecValTok{7}\NormalTok{))    }\CommentTok{# Returns the largest of given arguments or items in an iterable}
\end{Highlighting}
\end{Shaded}

\begin{verbatim}
## 28
\end{verbatim}

\begin{Shaded}
\begin{Highlighting}[]
\BuiltInTok{print}\NormalTok{(}\BuiltInTok{min}\NormalTok{(}\DecValTok{9}\NormalTok{,}\DecValTok{2}\NormalTok{,}\DecValTok{10}\NormalTok{,}\DecValTok{3}\NormalTok{,}\DecValTok{14}\NormalTok{,}\DecValTok{4}\NormalTok{,}\DecValTok{28}\NormalTok{,}\DecValTok{7}\NormalTok{))  }\CommentTok{# Returns the smallest of the given arguments or items in an) iterable}
\end{Highlighting}
\end{Shaded}

\begin{verbatim}
## 2
\end{verbatim}

\begin{Shaded}
\begin{Highlighting}[]
\BuiltInTok{print}\NormalTok{(}\BuiltInTok{pow}\NormalTok{(}\DecValTok{3}\NormalTok{,}\DecValTok{4}\NormalTok{))                   }\CommentTok{# Raises a number to a power}
\end{Highlighting}
\end{Shaded}

\begin{verbatim}
## 81
\end{verbatim}

\begin{Shaded}
\begin{Highlighting}[]
\BuiltInTok{print}\NormalTok{(}\BuiltInTok{round}\NormalTok{(}\FloatTok{3.1415962}\NormalTok{,}\DecValTok{3}\NormalTok{))         }\CommentTok{# Rounds a floating-point value}
\end{Highlighting}
\end{Shaded}

\begin{verbatim}
## 3.142
\end{verbatim}

\begin{Shaded}
\begin{Highlighting}[]
\BuiltInTok{print}\NormalTok{(}\BuiltInTok{sum}\NormalTok{([}\DecValTok{2}\NormalTok{,}\DecValTok{3}\NormalTok{,}\DecValTok{4}\NormalTok{,}\DecValTok{5}\NormalTok{,}\DecValTok{6}\NormalTok{]))       }
\end{Highlighting}
\end{Shaded}

\begin{verbatim}
## 20
\end{verbatim}

\begin{Shaded}
\begin{Highlighting}[]
\ImportTok{import}\NormalTok{ math                     }\CommentTok{# Import math package/module from Standard Library}
\BuiltInTok{print}\NormalTok{(math.sqrt(}\DecValTok{16}\NormalTok{))            }\CommentTok{# square root function}
\end{Highlighting}
\end{Shaded}

\begin{verbatim}
## 4.0
\end{verbatim}

\begin{Shaded}
\begin{Highlighting}[]
\BuiltInTok{print}\NormalTok{(math.exp(}\DecValTok{0}\NormalTok{))              }\CommentTok{# exponentiation function (powers of e)}
\end{Highlighting}
\end{Shaded}

\begin{verbatim}
## 1.0
\end{verbatim}

\begin{Shaded}
\begin{Highlighting}[]
\BuiltInTok{print}\NormalTok{(math.log10(}\DecValTok{5}\NormalTok{))            }\CommentTok{# base 10 logarithm}
\end{Highlighting}
\end{Shaded}

\begin{verbatim}
## 0.6989700043360189
\end{verbatim}

\begin{Shaded}
\begin{Highlighting}[]
\BuiltInTok{print}\NormalTok{(math.log(}\DecValTok{5}\NormalTok{))              }\CommentTok{# natural logarithm}
\end{Highlighting}
\end{Shaded}

\begin{verbatim}
## 1.6094379124341003
\end{verbatim}

A full list of the built-in (as opposed to Standard Library Funtions) is
displayed, from the python documentation, below.

\subsubsection{\texorpdfstring{\href{https://docs.python.org/3.6/library/functions.html}{Python
Built-in
Functions}}{Python Built-in Functions}}\label{python-built-in-functions}

\begin{longtable}[]{@{}lllll@{}}
\toprule
abs() & dict() & help() & min() & setattr()\tabularnewline
all() & dir() & hex() & next() & slice()\tabularnewline
any() & divmod() & id() & object() & sorted()\tabularnewline
ascii() & enumerate() & input() & oct() & staticmethod()\tabularnewline
bin() & eval() & int() & open() & str()\tabularnewline
bool() & exec() & isinstance() & ord() & sum()\tabularnewline
bytearray() & filter() & issubclass() & pow() & super()\tabularnewline
bytes() & float() & iter() & print() & tuple()\tabularnewline
callable() & format() & len() & property() & type()\tabularnewline
chr() & frozenset() & list() & range() & vars()\tabularnewline
classmethod() & getattr() & locals() & repr() & zip()\tabularnewline
compile() & globals() & map() & reversed() &
\textbf{import}()\tabularnewline
complex() & hasattr() & max() & round() &\tabularnewline
delattr() & hash() & memoryview() & set() &\tabularnewline
\bottomrule
\end{longtable}

R has a very large set of built-in functions available before the need
to load any libraries arises. The
\href{https://cran.r-project.org/doc/manuals/r-release/R-lang.html}{R
Language Reference} provides a comprehensive background on built in
fucntions as well as the rest of the language.

So R is centered around the mandate of being a complete mathematical
system, much like Matlab,(Octave or SciLab) which Python's manadate is
that of a full featured scripting and programming environment.

However, calling standard library modules in python puts it quite easily
and effectively on a even par with R. Going into more complex activites
requires thar both load or import fucther libraries. Like
\href{https://docs.python.org/3/library/index.html}{\textbf{Python's
Standard Library}}, \textbf{the R Core Library} shipping with the R
install is described in the
\href{https://cran.r-project.org/doc/manuals/r-release/fullrefman.pdf}{Full
R Reference Manaual} and covers a wide range of mathematical,
statistical and programming solutions.

We will go deeper into these and the other external Libraries and
modules in the next chapter, covering functional programming in more
detail. As well focused discussions on specific Data Science domains
form a block of Chapters in Part II of the book.

\chapter{Functional Programming with R and
Python}\label{functional-programming-with-r-and-python}

At this point we are ready to discuss the functional programming
paradigm in R and Python. The other important paradigm, Object-Oriented,
will be discussed in the following Chapter.

\section{Defining Functions}\label{defining-functions}

\subsection{R Scripting}\label{r-scripting-1}

\subsection{Python Scripting}\label{python-scripting-1}

\section{Calling and Using Functions}\label{calling-and-using-functions}

\subsection{R Scripting}\label{r-scripting-2}

\subsection{Python Scripting}\label{python-scripting-2}

\section{The Core or Standard
Libraries}\label{the-core-or-standard-libraries}

\subsection{R Scripting}\label{r-scripting-3}

\subsection{Python Scripting}\label{python-scripting-3}

\part{Data Science Topics in Python Compared to
R}\label{part-data-science-topics-in-python-compared-to-r}

\chapter{Using Probability Distributions with R and
Python}\label{using-probability-distributions-with-r-and-python}

At this point we are ready to discuss the functional programming
paradigm in R and Python. The other important paradigm, Object-Oriented,
will be discussed in the following Chapter.

\section{Basic Probability Issues}\label{basic-probability-issues}

\subsection{R Scripting}\label{r-scripting-4}

\subsection{Python Scripting}\label{python-scripting-4}

\section{Using the Distrbutions}\label{using-the-distrbutions}

\subsection{R Scripting}\label{r-scripting-5}

\subsection{Python Scripting}\label{python-scripting-5}

\section{Other Libraries with Probability and Statistical
Packages}\label{other-libraries-with-probability-and-statistical-packages}

\subsection{R Scripting}\label{r-scripting-6}

\subsection{Python Scripting}\label{python-scripting-6}

\chapter{Descriptive Statistics and Data
Exploration}\label{descriptive-statistics-and-data-exploration}

At this point we are ready to discuss the functional programming
paradigm in R and Python. The other important paradigm, Object-Oriented,
will be discussed in the following Chapter.

\section{Defining Functions}\label{defining-functions-1}

\subsection{R Scripting}\label{r-scripting-7}

\subsection{Python Scripting}\label{python-scripting-7}

\section{Calling and Using
Functions}\label{calling-and-using-functions-1}

\subsection{R Scripting}\label{r-scripting-8}

\subsection{Python Scripting}\label{python-scripting-8}

\section{The Core or Standard
Libraries}\label{the-core-or-standard-libraries-1}

\subsection{R Scripting}\label{r-scripting-9}

\subsection{Python Scripting}\label{python-scripting-9}

\chapter{Statistical Analysis and
Modeling}\label{statistical-analysis-and-modeling}

At this point we are ready to discuss the functional programming
paradigm in R and Python. The other important paradigm, Object-Oriented,
will be discussed in the following Chapter.

\section{Defining the Available
Functions}\label{defining-the-available-functions}

\subsection{R Scripting}\label{r-scripting-10}

\subsection{Python Scripting}\label{python-scripting-10}

\section{Calling and Using
Functions}\label{calling-and-using-functions-2}

\subsection{R Scripting}\label{r-scripting-11}

\subsection{Python Scripting}\label{python-scripting-11}

\chapter{Non-Stochastic Models}\label{non-stochastic-models}

Many mathematical models are deterministic. This includes mathematical
programming.

\chapter{Object-Oriented Programming with R and
Python}\label{object-oriented-programming-with-r-and-python}

At this point we are ready to discuss the functional programming
paradigm in R and Python. The other important paradigm, Object-Oriented,
will be discussed in the following Chapter.

\section{Defining Classes}\label{defining-classes}

\subsection{R Scripting}\label{r-scripting-12}

\subsection{Python Scripting}\label{python-scripting-12}

\section{Using Objects and Classes}\label{using-objects-and-classes}

\subsection{R Scripting}\label{r-scripting-13}

\subsection{Python Scripting}\label{python-scripting-13}

\chapter{Appendix 1}\label{appendix-1}

\section{Comparative Syntax for Programming Constructs of R and
Python}\label{comparative-syntax-for-programming-constructs-of-r-and-python}

As introduced in the text, the syntax of both R and python for the basic
programming constructs are reviewed there for reader convenience.

\begin{longtable}[]{@{}lll@{}}
\toprule
& R & Python\tabularnewline
\midrule
\endhead
Basic data types and structures & scalar == vector{[}0{]} &
scalars\tabularnewline
& vectors, (numerical, character, logical, factor) & (string, integer,
float)\tabularnewline
& arrays, == vectors (m x n) & arrays (m x n) of scalars\tabularnewline
& matrices, &\tabularnewline
& data frames, & dictionary,\tabularnewline
& and lists & and lists\tabularnewline
Operators & &\tabularnewline
conditionals & &\tabularnewline
Loops & &\tabularnewline
\bottomrule
\end{longtable}

\section{Extended Structures}\label{extended-structures}

\begin{center}\rule{0.5\linewidth}{\linethickness}\end{center}

\begin{center}\rule{0.5\linewidth}{\linethickness}\end{center}

\chapter{References}\label{references}

\begin{enumerate}
\def\labelenumi{\arabic{enumi}.}
\item
  Core R Team (Ed).,
  \href{https://cran.cnr.berkeley.edu/doc/manuals/r-release/fullrefman.pdf}{R
  Reference Index}, R Foundation for Statistical Computing, Vienna,
  2018.
\item
  Hellmann, Doug,
  \href{https://www.amazon.com/Python-Standard-Library-Example-Developers-ebook/dp/B072QZZDV7}{The
  Python 3 Standard Library by example}, Addison-Wesley, Boston MA,
  2017.
\item
  Python Software Foundation.,
  \href{https://docs.python.org/3.7/}{Python 3.7.1 documentation},
  Python Software Foundation, Wilmington, Delaware, 2018
\item
  RStudio Consortium,
  \href{https://support.rstudio.com/hc/en-us/categories/200035113-Documentation}{RStudio
  Documentation}, Boston, MA, 2018
\end{enumerate}


\end{document}
